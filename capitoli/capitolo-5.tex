% !TEX encoding = UTF-8
% !TEX TS-program = pdflatex
% !TEX root = ../tesi.tex

%**************************************************************
\chapter{Progettazione e codifica}
\label{cap:progettazione-codifica}
%**************************************************************

\intro{Breve introduzione al capitolo}\\

%**************************************************************
\section{Tecnologie e strumenti}
\label{sec:tecnologie-strumenti}

Di seguito viene data una panoramica delle tecnologie e strumenti utilizzati.

\subsection*{Javascript}
JavaScript è un linguaggio di programmazione orientato agli oggetti e agli eventi, comunemente utilizzato nella programmazione Web lato client per la creazione applicazioni web. L'intera applicazione è stata scritta con questo linguaggio.

\subsection*{React}
React è una libreria Javascript utilizzata per implementare interfacce utente (UI) lato frontend. React si basa sul concetto di component, idealmente è una libreria che permette di costruire i propri component come fossere degli elementi HTML del DOM per poi poterli riusare nell'intera applicazione.

\subsection*{Redux \& Redux RTK}
Redux è un contenitore dello stato per le applicazione Javascript. Viene usato per la gestione centralizzata dello stato delle applicazioni sviluppate in React Javascript. In particolare con la sua libreria Redux-Toolkit, permette una gestione dello stato semplice ed efficente.

\subsection*{API rest}
Descrizione Tecnologia 2

\subsection*{JSON}
Markup usato per la pubblicazione dei dati dell’API.

\subsection*{MUI}
Material UI è una libreria React open-source che permette di implementare i Google's Material Design. Essa comprende una collezione di componenti React precostruiti che possono essere facilmente adattati e messi in uso nella UI dell'applicazione. 

\subsection*{React Router}
React Router è la libreria standard per il routing in React. Questa libreria permette la navigazione tra le varie viste dell'applicazione , permette di gestire le URL, e mantenere la sincronizzazione tra URL e viste.

\subsection*{slack}
E uno strumento di collaborazione aziendale, che facilita la gestione delle comunicazioni all'interno del gruppo di lavoro.

\subsection*{Git}
Descrizione Tecnologia 2

\subsection*{Gitlab}
Descrizione Tecnologia 2

\subsection*{Jira}
Jira è uno strumento di gestione del lavoro

\subsection*{Docker}
Docker è un sistema che permette di facilitare il deployment delle applicazioni. 

\subsection*{VS Code}
IDE usato per lo sviluppo del progetto.

\subsection*{Jitsi}
Descrizione Tecnologia 2

\subsection*{Scrum} (???)
Descrizione Tecnologia 2

\subsection*{airbnb js guidelines} (???)
Descrizione Tecnologia 2

\subsection*{eslint} (???)
Descrizione Tecnologia 2

%**************************************************************
\section{Ciclo di vita del software}
\label{sec:ciclo-vita-software}

%**************************************************************
\section{Progettazione}
\label{sec:progettazione}

\subsubsection{Namespace 1} %**************************
Descrizione namespace 1.

\begin{namespacedesc}
    \classdesc{Classe 1}{Descrizione classe 1}
    \classdesc{Classe 2}{Descrizione classe 2}
\end{namespacedesc}


%**************************************************************
\section{Design Pattern utilizzati}

%**************************************************************
\section{Codifica}
