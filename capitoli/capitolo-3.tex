% !TEX encoding = UTF-8
% !TEX TS-program = pdflatex
% !TEX root = ../tesi.tex

%**************************************************************
\chapter{Descrizione dello stage}
\label{cap:descrizione-stage}
%**************************************************************

\intro{Breve introduzione al capitolo}\\
In questo capitolo viene descritto come si pianificato e svolto lo stage presso SAI, introducendo il progetto, considerando gli obiettivi e i possibili rischi.

%**************************************************************
\section{Introduzione al progetto}
Lo stage svolto presso SAI aveva come scopo la creazione di un prototipo di piattaforma web/desktop per il caricamento delle registrazioni audio effettuate nelle aule di tribunale, con l'aiuto di alcuni metadata prodotti durante le medesime registrazioni. In particolare l'applicazione da sviluppare è divisa in due parti backend e frontend (spiegare meglio back e front) che comunicano tra loro mediante API rest (spiegare API). Lo stage si è focalizzato sullo sviluppo della parte frontend, avanzando di pari passo con lo sviluppo del  backend realizzata da altri membri del team di sviluppo. Per la realizzazione del frontend si è scelto di utilizzare Javascript con i framework React e Redux. 

%**************************************************************
\section{Analisi preventiva dei rischi}

Nella fase di analisi iniziali si sono individuati i principali rischi a cui si poteva andare incontro e si è proceduto a definire le possibili soluzione per farne fronte.\\

\begin{risk}{Uso di nuove tecnologie}
    \riskdescription{le tecnologie proposte per la gestione e lo sviluppo del progetto erano per lo più nuove o scarsamente conosciute.}
    \risksolution{è stato previsto un periodo iniziale di studio e formazione su queste tecnologie.}
    \label{risk:new-tecnology} 
\end{risk}
\begin{risk}{Modalità di lavoro smart working}
    \riskdescription{lo stage è stato fatto completamente da remoto, e poteva portare ad una possibile mancanza di comunicazione e ad un incertezza nelle attività da svolgere.}
    \risksolution{il tutor aziendale si è reso disponibile a vari meeting nelle prime fasi del progetto e nella formazione iniziale, e rimanendo a disposizione per altri meeting  in caso di dubbi sulle attività da svolgere.}
    \label{risk:smart-working} 
\end{risk}

%**************************************************************
\section{Requisiti e obiettivi}
Si farà riferimento ai requisiti secondo le seguenti notazioni:
\begin{itemize}
\item O per i requisiti obbligatori, vincolanti in quanto obiettivo primario richiesto dal committente;
\item D per i requisiti desiderabili, non vincolanti o strettamente necessari, ma dal riconoscibile valore aggiunto;
\item F per i requisiti facoltativi, rappresentanti valore aggiunto non strettamente competitivo.
Le sigle precedentemente indicate saranno seguite da una coppia sequenziale di numeri, identificativo del requisito.
\end{itemize}
\begin{center}
\renewcommand{\arraystretch}{1.8} %aumento ampiezza righe
\begin{table}
\begin{tabular}{ |c|p{30em}| }
\hline
O01 & autenticazione mediante server remoto \\
\hline
O02 & lettura dati da CD \\
\hline
O03 & precompilazione di form con i dati caricati da CD \\
\hline
O04 & editing dei dati del form \\
\hline
D01 & upload dei dati verso i sistemi esterni \\
\hline
D02 & test di unità esaustivi \\
\hline
F01 & possibilità di ascoltare le registrazioni \\
\hline
F02 & possibilità di modificare i dati mediante interazioni evolute (per es. drag-n-drop) \\
\hline
F03 & realizzazione di un’applicazione desktop con Electron \\
\hline
F04 & compilazione multipiattaforma dell’applicazione desktop \\
\hline
\end{tabular}
\vskip .5cm
\caption{Tabella dei Requisiti}
\end{table}
\end{center}
%**************************************************************
\section{Pianificazione}