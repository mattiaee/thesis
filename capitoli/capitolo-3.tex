% !TEX encoding = UTF-8
% !TEX TS-program = pdflatex
% !TEX root = ../tesi.tex

%**************************************************************
\chapter{Descrizione dello stage}
\label{cap:descrizione-stage}
%**************************************************************

\intro{Breve introduzione al capitolo}\\

%**************************************************************
\section{Introduzione al progetto}

%**************************************************************
\section{Analisi preventiva dei rischi}

Durante la fase di analisi iniziale sono stati individuati alcuni possibili rischi a cui si potrà andare incontro.
Si è quindi proceduto a elaborare delle possibili soluzioni per far fronte a tali rischi.\\

\begin{risk}{Performance del simulatore hardware}
    \riskdescription{le performance del simulatore hardware e la comunicazione con questo potrebbero risultare lenti o non abbastanza buoni da causare il fallimento dei test}
    \risksolution{coinvolgimento del responsabile a capo del progetto relativo il simulatore hardware}
    \label{risk:hardware-simulator} 
\end{risk}

%**************************************************************
\section{Requisiti e obiettivi}
Si farà riferimento ai requisiti secondo le seguenti notazioni:
\begin{itemize}
\item O per i requisiti obbligatori, vincolanti in quanto obiettivo primario richiesto dal committente;
\item D per i requisiti desiderabili, non vincolanti o strettamente necessari, ma dal riconoscibile valore aggiunto;
\item F per i requisiti facoltativi, rappresentanti valore aggiunto non strettamente competitivo.
Le sigle precedentemente indicate saranno seguite da una coppia sequenziale di numeri, identificativo del requisito.
\end{itemize}
\begin{center}
\renewcommand{\arraystretch}{1.8} %aumento ampiezza righe
\begin{table}
\begin{tabular}{ |c|p{30em}| }
\hline
O01 & autenticazione mediante server remoto \\
\hline
O02 & lettura dati da CD \\
\hline
O03 & precompilazione di form con i dati caricati da CD \\
\hline
O04 & editing dei dati del form \\
\hline
D01 & upload dei dati verso i sistemi esterni \\
\hline
D02 & test di unità esaustivi \\
\hline
F01 & possibilità di ascoltare le registrazioni \\
\hline
F02 & possibilità di modificare i dati mediante interazioni evolute (per es. drag-n-drop) \\
\hline
F03 & realizzazione di un’applicazione desktop con Electron \\
\hline
F04 & compilazione multipiattaforma dell’applicazione desktop \\
\hline
\end{tabular}
\vskip .5cm
\caption{Tabella dei Requisiti}
\end{table}
\end{center}
%**************************************************************
\section{Pianificazione}